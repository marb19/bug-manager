%Empieza configuracion
\setstretch{1.0}
\titleformat{\chapter}{\Huge\bfseries}{\thechapter}{0 pt}{\rule{340 pt}{3 pt}\\}
\titlespacing{\chapter}{100 pt}{-25 pt}{40 pt}[10 pt]	
\pagestyle{fancy}
\fancyhead[RO,RE]{\thepage}
\fancyfoot[CO,CE]{}
%Termina configuracion

\chapter*{Resumen}
\addcontentsline{toc}{chapter}{Resumen}
\setstretch{1.5} %Regresa el interlineado a 1.5


\normalsize
\noindent Este trabajo de Tesis presenta el sistema de administraci'on que se llamar'a "Bug Manager" (BM). Con este sistema se demuestra como la implementaci'on de conceptos simples y claves de calidad dentro del proceso de desarrollo de software en empresas medianas y peque'nas, ayudan a mejorar sustancialmente los tiempos de entrega, la calidad final del producto y la satisfacci'on de los clientes. Algunos de estos conceptos son:

\begin{itemize}
	\item Registro de actividades con su esfuerzo y tama'no dedicados;
	\item Registro y seguimiento de defectos encontrados;
	\item Realizaci'on de revisiones de c'odigo;
	\item Medici'on de la productividad personal y global de la empresa.
\end{itemize}

Con estos conceptos en mente, se propone la creaci'on del BM, el cual ser'a utilizado por cualquier empresa que busque implementar t'ecnicas de control de calidad en su proceso de desarrollo de software. Este producto va enfocado inicialmente a peque'nas y medianas empresas, principalmente mexicanas. Esto 'ultimo en virtud de que es el mercado preferente al cual se pretende llegar.

El BM permitir'a que las empresas tener control sobre el avance y la calidad de sus proyectos, por medio de las siguientes estrategias:

\begin{itemize}
	\item Gu'ia en la elaboraci'on del plan de calidad.
	\item Definici'on del ciclo de vida y actividades de desarrollo.
	\item Registro y seguimiento de actividades de aseguramiento de calidad.
	\item Registro y seguimiento de defectos.
	\item Generaci'on de estad'isticas personales, por proyecto, por equipo y por empresa.
\end{itemize}

Finalmente, durante el desarrollo del BM se pondr'an en pr'actica las actividades de calidad mencionadas, y se har'a un an'alisis del costo de la calidad para comprobar la efectividad de estas actividades en la mejora del proceso de desarrollo de software.

\clearpage