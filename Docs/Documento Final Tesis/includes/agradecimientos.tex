%Empieza configuracion
\setstretch{1.0}
\titleformat{\chapter}{\Huge\bfseries}{\thechapter}{0 pt}{\rule{340 pt}{3 pt}\\}
\titlespacing{\chapter}{100 pt}{-25 pt}{40 pt}[10 pt]	
\pagestyle{fancy}
\fancyhead[RO,RE]{\thepage}
\fancyfoot[CO,CE]{}
%Termina configuracion

\chapter*{Agradecimientos}
\addcontentsline{toc}{chapter}{Agradecimientos}
\setstretch{1.5} %Regresa el interlineado a 1.5

\normalsize
\noindent 
Quiz'a no estaba en plenitud de mis sentidos en el momento que decid'i iniciar una Maestr'ia en Ciencias de la Computaci'on. Definitivamente se necesita un poco de locura, y un tanto m'as de agallas para embarcarse en una maestr'ia con enfoque cient'ifico, en la que tengas que crear y defender una tesis para obtener el grado. Esto se pone un poco m'as interesante si le agregamos el hecho de tener un trabajo de tiempo completo y otro de medio tiempo mientras se cursa la maestr'ia. Cuando se ve esta perspectiva de una forma m'as general, se comprende que esto no lo hubiera logrado yo solo, y que, de alguna manera u otra, se ha llegado al objetivo gracias al apoyo de varias personas a las que quisiera agradecer personalmente y en esta dedicatoria.

Primero que nada quiero agradecer al Doctor 'Oscar Mondrag'on, el director de esta tesis, por todo el apoyo que me ha brindado durante este trabajo de un a'no. El decidi'o desinteresadamente apoyarme, y me ayud'o a transformar una serie de ideas en un producto completo y palpable. Con sus conocimientos y experiencias personales y profesionales, fue una gran gu'ia durante toda esta traves'ia.

Tambi'en es muy importante mencionar a mis compa'neros que elaboraron junto conmigo el sistema que es la base del estudio. Eduardo Campos y Humberto Garc'ia, aparte de ser dos estudiantes sobresalientes, son a'un mejores como personas, siendo destacados por tener valores como la responsabilidad, la honestidad, el trabajo duro y un sentido de compromiso como pocas personas en nuestro pa'is.

Quisiera agradecer tambi'en a mis padres y hermanos, Antonio Rangel, Patricia Bocardo, Julia Rangel, Bryan Rangel y 'Angel Rangel, por todo el apoyo que me brindan en cada uno de mis proyectos, y aunque no hayan cooperado de forma directa con el trabajo, su apoyo moral siempre ha sido muy importante para lograr lo que me propongo.

Cuento con la suerte de tener muchos y excepcionales amigos, as'i que no quer'ia perder la oportunidad de mencionarlos, ya que ellos hacen que mi vida sea m'as plena y est'e llena de experiencias divertidas. Agradezco mucho a: Guille Suro, Naiv Soto, Jafet Rodr'iguez, Humberto Garc'ia,  H'ector Sol'is, Juli'an Hern'andez, Leonel Haro y Joaqu'in Soto; por todos los buenos momentos, las risas y las aventuras que hemos vivido.

Finalmente quisiera agradecer de forma especial a Alejandro V'azquez. 'El nos brind'o apoyo t'ecnico y conocimientos durante la elaboraci'on del sistema, todo de forma desinteresada, sin recibir nada a cambio, y siempre con una actitud de ayuda y servicio.

\clearpage